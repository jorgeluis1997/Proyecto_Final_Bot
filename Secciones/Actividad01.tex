\section{Proyecto Final de Unidad -  Sistema Recomendacion de Libros} 
\begin{flushleft}

\begin{itemize}
\textbf{1.Antecedentes o Introducción}\\


La palabra Bot proviene de la palabra Robot y es la forma como se le denomina en el lenguaje tecnológico que realiza un procedimiento de manera automática en distintos contextos.
Es un concepto que existe hace casi 50 años, pero con las nuevas herramientas y posibilidades que brinda la época actual, está ganando popularidad dada las diversas aplicaciones que tiene en el mercado.\textbf{ }\\
\textbf{ }\\
\textbf{ }\\
Existen varios tipos de Bots, acá nombraremos algunos de los más conocidos: \textbf{ }\\
\textbf{ }\\
•	Bots de seguimiento en redes sociales (Following Bots)\textbf{ }\\
•	Bots de tráfico para sitios web (Traffic Bots)\textbf{ }\\
•	Bots Informativos\textbf{ }\\
•	Bots de conversación (ChatBots)\textbf{ }\\
\textbf{ }\\
El servicio al cliente en redes sociales está cada día más en tendencia, tenemos claro que es más fácil avisar de la falla en algún servicio en redes sociales que llamar y esperar minutos a ser atendido por una operadora.\textbf{ }\\
\textbf{ }\\
Es biológicamente imposible que un ejecutivo telefónico pueda atender a más de una persona a la vez por lo que las redes sociales y los canales de comunicación de Internet, tales como chat, correo electrónico, o servicios de mensajería móvil son el conducto ideal para atender a más clientes de forma simultánea, ya que un único ejecutivo puede atender 5 o más personas a la vez.\textbf{ }\\
\textbf{ }\\
Si a esta ventaja en atención al cliente le agregaremos bots de primer nivel podemos atender hasta 4 veces más personas que por teléfono con los beneficios adicionales de operar 24/7 todos los días del año.\textbf{ }\\
La idea de tener un bot es complementar la atención humana, haciéndola más eficiente. Apoyar con preguntas frecuentes, consultas simples o tomar datos de manera automática son algunos de los casos más comunes donde se puede aplicar esta tecnología.\textbf{ }\\
\textbf{ }\\
La finalidad de tener una biblioteca con información actualizada y brindar información rápida diferentes medios bibliográficos de la biblioteca, además de poder realizar compras en la pagina web.\textbf{ }\\
El Sistema bibliotecario tiene como objetivo proveer servicios de consulta, pedidos, compras y anulaciones de solicitudes de libros.
\textbf{ }\\
\textbf{ }\\
Para la creación de este sistema para la administración de la información de los distintos libros de la biblioteca, se utilizará la clasificación de los libros así como los diferentes medios que posee la biblioteca como lo son: revistas, libros.
 \textbf{ }\\
\textbf{ }\\
Este sistema será de mucha utilidad para ubicar un libro y otros medios de la biblioteca rápidamente, nos facilitará conocer el status de los libros, la adquisición de nuevos libros y los procesos técnicos por ejemplo catalogación y clasificación de los ejemplares.
\textbf{ }\\
\textbf{ }\\
La finalidad principal de un proyecto de este tipo es la generación, administración y disposición de conocimiento para una comunidad determinada.
Los beneficios académicos es automatizar estos procesos con un sistema de biblioteca que ofrece recomendaciones y compras de libros haciendo la tarea más sencilla para los usuarios.
\textbf{ }\\
\textbf{ }\\
Con la implementación de este sistema se obtendrá la información al instante de los libros, revista, editoriales entre otros. Se podrá obtener una lista de todos los libros en stock, editoriales, etc., y buscar en cualquier momento en base a varios reglas de filtrado. Se podrá organizar la biblioteca por editoriales, autores entre otros.


